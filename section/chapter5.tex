\chapter{Conclusion}
brief of conclusion

\section{Conclusion of Problems}
Tell about solving the problem

\section{Conclusion of Method}
Tell about solving using method

\section{Conclusion of Experiment}
Tell about solving in the experiment

\section{Conclusion of Result}
tell about result for purpose of this research.
<<<<<<< HEAD
\section{Puad Hamdani/1164084}
\subsection{Teori}
\begin{enumerate}
\item Vektorisasi Kata-Kata
\begin{itemize}
\item Penjelasan :
\par Alasan mengapa kata-kata harus dilakukan vektorisasi yaitu dikarenakan mesin hanya mampu membaca data dengan bentuk angka. Berdasarkan hal tersebut,diperlukan vektorisasi kata atau mengubah kata menjadi bentuk vektor agar mesin  paham apa yang kita maksudkan dan dapat memproses perintah dengan benar.
\item Mengapa Dimensi Dari Vektor Dataset Google Bisa Mencapai 300


\item  Penjelasan :
\par Dimensi dari Vektor Dataset Google Mencapai 300 karena  pada masing-masing objek yang terdapat pada dataset akan memiliki identitasnya tersendiri, selain itu juga untuk nilai dalam vektor 300 dimensi yang terkait dalam sebua kata "dioptimalkan" dalam  berbagai hal untuk menangkap aspek yang berbeda dari makna dan penggunaan kata itu. secara singkatnya terdapat ada lebih dari 3 miliar kata-kata dan kalimat yang tidak mungkin disimpan dalam 1 diemensi vektor, lalu disimpan menjadi 300 dimensi vektor untuk mengatasi yang namanya kegagalan memori
\end{itemize}
\item Konsep Vektorisasi Untuk Kata
\begin{itemize}
\item  Penjelasan :
\par Konsep untuk vektorisasi kata sebenarnya sama dengan ketika dilakukan input suatu kata pada mesin pencarian. Kemudian untuk hasilnya akan mengeluarkan ( berupa ) referensi mengenai kata tersebut. Jadi data kata tersebut didapatkan dari hasil pengolahan pada kalimat-kalimat sebelumnya yang telah diolah. Contoh sederhananya pada kalimat berikut ( Please click the alarm icon for more notifications about my channel ), pada kalimat tersebut terdapat konteks yakni channel, kata tersebut akan dijadikan data latih untuk mesin yang akan dipelajari dan diproses. Jadi ketika kita inputkan kata channel, maka mesin akan menampilkan keterkaitannya dengan kata tersebut sehingga akan lebih efisien dan lebih mudah.
\end{itemize}
\item Konsep Vektorisasi Untuk Dokumen
\begin{itemize}
\item  Penjelasan :
\par Untuk vektorisasi dokumen sebenarnya terbilang sama dengan konsep vektorisasi kata, yang membedakan hanya pada proses awalnya ( pada eksekusi awal ). Untuk vektorisasi dokumen ini, mesin akan membaca semua kalimat yang terdapat pada dokumen tersebut, kemudian kalimat yang terdapat pada dokumen tersebut akan di pecah menjadi kata-kata. Seperti itulah konsep vektorisasi dokumen.
\end{itemize}
\item Pengertian Mean Dan Standar Devisiasi
\begin{itemize}
\item  Pengertian Mean :
\par Mean merupakan nilai rata-rata dari suatu data. Mean sendiri dapat dicari dengan cara membagi jumlah data dengan banyak data sehingga diperoleh lah nilai rata-rata dari suatu data yang dicari / tersebut. 

\item  Pengertian Standar Devisiasi :
\par Untuk standar deviasi sendiri merupakan sebuah teknik statistik yang digunakan dalam menjelaskan homogenitas kelompok ataupun dapat diartikan dengan nilai statistik dimana dimanfaatkan untuk menentukan bagaimana sebaran data dalam sampel, serta seberapa dekat titik data individu ke mean atau rata-rata nilai sampel yang ada. 
\end{itemize}
\item Penjelasan Skip-gram
\begin{itemize}
\item  Penjelasan :
\par Sebuah  teknik yang digunakan di area speech processing, dimana n-gram yang dibentuk kemudian ditambahkan juga dengan tindakan “skip” pada token-tokennya. 
\par Untuk membentuk k-skip-n-grams, ada dua nilai yang harus didefinisikan, dimana kedua nilai tersebut yaitu k (jumlah kata yang di-skip) dan n (banyak kata dalam n-gram, e.g. bigram (2-gram), trigram (3-gram), dll.).
\end{itemize}

\end{enumerate}

\chapter{Experiment and Result}
brief of experiment and result.
\section{Experiment}
Please tell how the experiment conducted from method.

\section{Result}
Please provide the result of experiment

\section{Mhd Zulfikar Akram Nastuion / 1164081}
\subsection{Teori}
\begin{enumerate}
\item Klasifikasi Teks
\par
Klasifikasi teks adalah sebuah proses untuk menempatkan dokumen teks ke dalam suatu kategori berdasarkan isi dari teks tersebut. 
\item Kenapa klasifikasi bunga tidak bisa menggunakan machine learning
\par
Klasifikasi bunga tidak dapat digunakan pada machine learning karena terdapat masalah, yaitu apabila kita memasukkan suatu inputan yang sama tetapi outputnya berbeda.
\item  Teknik pembelajaran mesin pada teks pada kata-kata yang digunakan di Youtube
\par
Penggunaan machine learning pada Youtube contohnya ada pada saat kita sedang melakukan pencarian judul, nama dan lainnya maka youtube akan menampilkan apa yang kita cari, kemudian saat kita sedang melihat youtube, dibagian kanan video terlihat ada tampilan video yang berkaitan dengan apa yang kita cari atau kita ketikkan di tempat pencarian terbebut.
\item Vectorisasi Data
\par
Vektorisasi data adalah sebuah proses pembagian dan pemecahan data kemudian dilakukan perhitungan datanya.
\item Bag of Words
\par
Bag of words adalah sebuah proses penyederhanaan yang digunakan dalam pengambilan informasi.
\item TF-IDF
\par
TF-IDF adalah sebuah metode untuk menghitung beberapa kata yang muncul dari setiap kata yang paling umum digunakan.

\end{enumerate}

==========

\section{Puad Hamdani / 116408}
\subsection{Teori}
\begin{enumerate}
\item Klasifikasi Teks
\par
Klasifikasi teks adalah sebuah proses untuk menempatkan dokumen teks ke dalam suatu kategori berdasarkan isi dari teks tersebut. 
\item Kenapa klasifikasi bunga tidak bisa menggunakan machine learning
\par
Klasifikasi bunga tidak dapat digunakan pada machine learning karena terdapat masalah, yaitu apabila kita memasukkan suatu inputan yang sama tetapi outputnya berbeda.
\item  Teknik pembelajaran mesin pada teks pada kata-kata yang digunakan di Youtube
\par
Penggunaan machine learning pada Youtube contohnya ada pada saat kita sedang melakukan pencarian judul, nama dan lainnya maka youtube akan menampilkan apa yang kita cari, kemudian saat kita sedang melihat youtube, dibagian kanan video terlihat ada tampilan video yang berkaitan dengan apa yang kita cari atau kita ketikkan di tempat pencarian terbebut.
\item Vectorisasi Data
\par
Vektorisasi data adalah sebuah proses pembagian dan pemecahan data kemudian dilakukan perhitungan datanya.
\item Bag of Words
\par
Bag of words adalah sebuah proses penyederhanaan yang digunakan dalam pengambilan informasi.
\item TF-IDF
\par
TF-IDF adalah sebuah metode untuk menghitung beberapa kata yang muncul dari setiap kata yang paling umum digunakan.

\end{enumerate}

\subsection{Prakteki}
\begin{enumerate}
\item aplikasi sederhana menggunakan pandas
\par


\end{enumerate}
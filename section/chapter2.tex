\chapter{Related Works}

Your related works, and your purpose and contribution which must be different as below.


\section{puad hamdani/1164084}
\subsection{binary classification }
\begin{enumerate}
\item Output aktual dari banyak algoritma binary classification adalah skor prediksi. Skor menunjukkan kepastian sistem bahwa pengamatan yang diberikan adalah milik kelas positif. Untuk membuat keputusan tentang apakah pengamatan harus diklasifikasikan sebagai positif atau negatif
\end{enumerate}

\subsection{supervised learning dan unsupervised learning}
\begin{enumerate}
\item Supervised learning adalah sebuah pendekatan dimana sudah terdapat data yang dilatih, dan terdapat variable yang di targetkan sehingga tujuan dari pendekatan ini adalah mengelompokkan suatu data ke data yang sudah ada.


\item Unsupervised learning adalah istilah yang digunakan untuk pembelajaran bahasa Ibrani, yang terkait dengan pembelajaran tanpa guru, juga dikenal sebagai organisasi mandiri dan metode pemodelan kepadatan probabilitas input. Unsupervised learning tidak memiliki data latih, sehingga dari data yang ada, kita mengelompokan data tersebut menjadi dua bagian atau tiga bagian dan seterusnya.
\end{enumerate}

\subsection{evaluasi dan akurasi }
\begin{enumerate}
\item Evaluasi merupakan suatu proses identifikasi untuk mengukur atau menilai apakah suatu kegiatan/program yang dilaksanakan sesuai dengan perencanaan atau tujuan yang ingin dicapai.
Akurasi merupakan tingkat kedekatan pengukuran kuantitas terhadap nilai yang sebenarnya. Kepresisian dari suatu sistem pengukuran
\end{enumerate}

\subsection{ bagaimana cara membuat dan membaca confusion matrix, buat confusion matrix }
\begin{enumerate}
\item Cara membuat dan membaca confusion matrix :
\begin{itemize}
\item 1)	Tentukan pokok permasalahan dan atributanya
\item 2)	Buat pohon keputusan
\item 3)	Lalu data testingnya
\item 4)	Lalu mencari nilai a, b, c, dan d. Semisal a = 5, b = 1, c = 1, dan d = 3.
\item 5)	Selanjutnya mencari nilai recall, precision, accuracy, serta dan error rate.
\end{itemize}
\item Berikut adalah contoh dari confusion matrix :
\begin{itemize}
\item Recall =3/(1+3) = 0,75
\item Precision = 3/(1+3) = 0,75
\item Accuracy =(5+3)/(5+1+1+3) = 0,8
\item Error Rate =(1+1)/(5+1+1+3) = 0,2
\end{itemize}
\end{enumerate}

\subsection{bagaimana K-fold cross validation bekerja dengan gambar ilustrasi}
\begin{enumerate}
\item Cara kerja K-fold cross validation :
\begin{itemize}
\item 1)	Total instance dibagi menjadi N bagian.
\item 2)	Fold yang pertama adalah bagian pertama menjadi data uji (testing data) dan sisanya menjadi training data.
\item 3)	Lalu hitung akurasi berdasarkan porsi data tersebut dengan menggunakan persamaan.
\item 4)	Fold yang ke dua adalah bagian ke dua menjadi data uji (testing data) dan sisanya training data. 
\item 5)	Kemudian hitung akurasi berdasarkan porsi data tersebut.
\item 6)	Dan seterusnya hingga habis mencapai fold ke-K.
\item 7)	Terakhir hitung rata-rata akurasi K buah.
\end{itemize}
\end{enumerate}

\subsection{decision tree}
\begin{enumerate}
\item Decision tree merupakan model prediksi menggunakan struktur pohon atau struktur berhirarki.
\end{enumerate}

\subsection{Information Gain}
\begin{enumerate}
\item Information gain. Metode tersebut akan melakukan proses komputasi untuk mendapatkan atribut-atribut yang paling berpengaruh terhadap dataset
\end{enumerate}
\section{Same Topics}
Cite every latest journal with same topic
\subsection{Topic 1}
cite for first topic

\subsection{Topic 2}
if you have two topics you can include here to


\section{Same Method}
write and cite latest journal with same method

\subsection{Method 1}
cite and paraphrase method 1

\subsection{Method 2}
cite and paraphrase method 2 if you have more method please add new subsection.

 

